% PositionalEncoder_Ex.tex
% Worked numeric example for CircuitHistoryEncoder + PositionalEncoding
% Generated to accompany transformer/PositionalEncoding.py
\documentclass{article}
\usepackage{amsmath}
\usepackage{amsfonts}
\usepackage{amssymb}
\usepackage{bm}
\usepackage{graphicx}
\usepackage{hyperref}
\begin{document}
\title{Worked example: CircuitHistoryEncoder and Positional Encoding}
\author{}
\date{}
\maketitle

This document walks through a tiny, concrete example of how the \texttt{CircuitHistoryEncoder} and the sinusoidal \texttt{PositionalEncoding} (as implemented in \texttt{transformer/PositionalEncoding.py}) transform a short sequence of circuit actions into position-aware embeddings suitable for a transformer.

\section*{Setup and conventions}
We use the following conventions consistent with the code:
\begin{itemize}
  \item Embedding dimension (model width): $d_{model} = 8$.
  \item Token types: 0=pad, 1=input, 2=constant, 3=add, 4=multiply.
  \item Node index embedding vocabulary size: 100 (indices 0..99).
  \item The transformer expects input shaped as $(\text{seq\_len}, \text{batch}, d_{model})$.
\end{itemize}

\section*{Actions and tokens}
Consider a tiny circuit with three actions (three nodes):
\begin{align*}
\text{actions} = [&\; (\text{"input"}, -, -), \;
                  (\text{"input"}, -, -), \;
                  (\text{"add"}, 0, 1) \;]
\end{align*}

Using \texttt{encode\_circuit\_actions}, these become the token list (sequence length $S=3$):
\begin{align*}
\tau_0 &= \{\text{type}:1,\; \text{value}:0,\; \text{node\_idx}:0\},\\
\tau_1 &= \{\text{type}:1,\; \text{value}:1,\; \text{node\_idx}:1\},\\
\tau_2 &= \{\text{type}:3,\; \text{value}:0,\; \text{node\_idx}:2,\; \text{input1}:0,\; \text{input2}:1\}.
\end{align*}

\section*{Embedding step}
Each token yields three embeddings which are summed elementwise to produce the final token embedding:
\begin{itemize}
  \item $E_{type}(\tau)$: from a learnable \texttt{nn.Embedding(5, d\_model)} using the token \texttt{type}.
  \item $E_{value}(\tau)$: from a learnable linear layer \texttt{nn.Linear(1, d\_model)} applied to the scalar \texttt{value}.
  \item $E_{node}(\tau)$: from a learnable \texttt{nn.Embedding(100, d\_model)} using \texttt{node\_idx}.
\end{itemize}

\subsection*{Concrete computed embeddings}
The encoder output (before batching) is the time-first matrix $X$ with shape $(S, d_{model})$ where $S=3$.  
We truncate values to two decimals for readability:
\begin{align*}
X &\approx
\begin{bmatrix}
1.07 & -0.16 & -0.86 & -3.53 & -0.68 & -1.20 & 0.58 & -0.14 \\
-0.46 & -0.15 & 0.94 & -2.41 & -3.41 & 0.41 & 1.36 & -0.30 \\
-0.56 & -2.16 & 0.28 & 3.31 & 2.04 & -0.51 & -1.87 & 0.05
\end{bmatrix}
\end{align*}

\section*{Batch dimension and positional encoding}
We add a batch dimension ($B=1$):
\begin{align*}
X_{batched} &\in \mathbb{R}^{3 \times 1 \times 8}, \quad
X_{batched} = \begin{bmatrix} \mathbf{x}_0 \\ \mathbf{x}_1 \\ \mathbf{x}_2 \end{bmatrix}.
\end{align*}

The sinusoidal positional encoding matrix $P$ is defined by:
\begin{align*}
P_{pos,2i} &= \sin\!\left( \frac{pos}{10000^{2i/d_{model}}} \right), &
P_{pos,2i+1} &= \cos\!\left( \frac{pos}{10000^{2i/d_{model}}} \right).
\end{align*}

For $d_{model}=8$ and positions $0,1,2$, rounded values are:
\begin{align*}
P_{0,:} &\approx [0, 1, 0, 1, 0, 1, 0, 1],\\
P_{1,:} &\approx [0.84, 0.54, 0.10, 0.99, 0.01, 1.00, 0.00, 1.00],\\
P_{2,:} &\approx [0.91, -0.42, 0.20, 0.98, 0.02, 1.00, 0.00, 1.00].
\end{align*}

Applying $Y = X_{batched} + P_{0:3, :}$ gives:
\begin{align*}
Y &\approx
\begin{bmatrix}
1.07 & 0.84 & -0.86 & -2.53 & -0.68 & -0.20 & 0.58 & 0.86 \\
0.38 & 0.39 & 1.04 & -1.42 & -3.40 & 1.41 & 1.36 & 0.70 \\
0.35 & -2.57 & 0.48 & 4.29 & 2.06 & 0.49 & -1.87 & 1.05
\end{bmatrix}
\end{align*}

\section*{Remarks}
\begin{itemize}
  \item The positional encoding is deterministic and fixed (not learned).
  \item The encoder sums three information sources: type, value, and node identity.
  \item For batching circuits of varying lengths, pad to a common $S$ and use attention masks.
  \item Avoid fixing device tensors at construction; use \texttt{module.to(device)}.
\end{itemize}

\section*{Short example: encoding $x_0 + x_1 + 1$}
Below is a concise walkthrough of how the expression $x_0 + x_1 + 1$ is turned into the network input:
\begin{enumerate}
  \item Linearize the computation as actions (one action per node), e.g.:
    \begin{align*}
    (&\text{"input"}, -, -) \quad (\text{node }0),\\
    (&\text{"input"}, -, -) \quad (\text{node }1),\\
    (&\text{"constant"}, -, -) \quad (\text{node }2),\\
    (&\text{"add"}, 0, 1) \quad (\text{node }3 \; = x_0 + x_1),\\
    (&\text{"add"}, 3, 2) \quad (\text{node }4 \; = (x_0 + x_1) + 1).
    \end{align*}
  \item Tokens: each action becomes a token dictionary with fields \texttt{type}, \texttt{value}, and \texttt{node\_idx} (and inputs for ops). Example token sequence: 
    \[\tau_0, \tau_1, \tau_2, \tau_3, \tau_4\].
  \item Embedding: for each token we compute
    \[x_i = E_{type}(\tau_i) + E_{value}(\tau_i) + E_{node}(\tau_i)\]
    resulting in a time-first matrix $X \in \mathbb{R}^{S\times d_{model}}$ (here $S=5$).
  \item Batch and add positional encodings: convert to $X_{batched}\in\mathbb{R}^{S\times 1\times d_{model}}$ and compute
    \[Y = X_{batched} + P_{0:S,:}\]
    which is the final input to the transformer.
\end{enumerate}

\subsection*{What the transformer does}
After receiving the positional-encoded input $Y$, the transformer applies stacked layers of multi-head self-attention and positionwise feed-forward networks. The self-attention layers let each position (token) attend to other positions in the sequence (so the model can combine input tokens and constants according to the computation graph), and the feed-forward layers mix and transform these combined features.

Typical outputs depend on the task: for generation or prediction you may decode from the final sequence representations (e.g., project the last token or all tokens to logits over actions or values); for classification/regression you might pool across positions (mean, attention-pool, or take a special token) and apply a final linear head to produce the desired scalar or categorical outputs.

\section*{Beginner-friendly: switching to a one-hot encoder}
We will replace the current sum-of-embeddings input with a simple one-hot style encoder. Here's a plain-language explanation a newcomer can follow:

\begin{itemize}
  \item A "one-hot" vector is just a long row of zeros with a single 1 in the position that represents some choice. For example, if we have 5 token types and the token is an "input", the type one-hot looks like [0,1,0,0,0].
  \item For each token we build a few small one-hot vectors (one for the token type, one for the node index, and one for the value bucket). We glue those small rows together into one longer row. That long row is the token's feature vector.
  \item A learned linear layer then squashes this long one-hot row down to the transformer's working size (the \(d_{model}\) that the transformer expects). This is similar to how the current embeddings work, but more explicit about each field.
  \item After this projection we add the same positional encoding and pass the result into the transformer unchanged.
\end{itemize}

\paragraph{Why this is helpful for beginners:}
\begin{itemize}
  \item It's very explicit — every bit of the input corresponds to a clear, human-readable field (type, node id, value bucket).
  \item It's easy to convert model outputs back into readable predictions (look at which position in the one-hot is most likely).
  \item It's straightforward to implement and reason about: build one-hot, project, add position, run transformer.
\end{itemize}

Practical note: if the node index or value spaces are very large, the one-hot vector gets large. In that case we can still use the same idea but implement it efficiently with small embedding tables for node indices and types (those are equivalent but more memory-friendly).

\end{document}
